\chapter{General section}

\section{Summary of the research program}

The proposed \thiscrc{} aims to investigate aspects of \blindtext
The central question we aim to
investigate in \thiscrc{} is:

\begin{description}
\item[Q:] \textbf{What is Love?} 
\end{description}

\blindtext \textcite{test1}

The goals of \thiscrc{} are threefold:
\begin{enumerate}
\item One
\item Two
\item Three
\end{enumerate}

\section{Research program and long-term goals}

\subsection{Introduction}
\label{sec:intr-regist-regist}

\blindtext \textcite{buch1}

\subsection{Background on relevant strands of prior research }
\label{sec:backgr-relev-strands}

\blindtext \textcite{teil1}

\subsection{Core issues for the study of the field}
\label{sec:core-issues-crc}


\subsection{Research areas}
\label{sec:research-areas}

Recall now our central question Q from above: 

\begin{description}
\item[Q:] \textbf{What is Love?} 
\end{description}

We plan to pursue Q in three research areas, A: Unicorns,
B: Hedgehogs, and C: Deers. These areas
provide a useful structure \blindtext

\subsubsection{Area A: Unicorns}

\blindtext
\subsubsection{Area B: Hedgehogs}

\blindtext

\subsubsection{Area C: Deers}

\blindtext

\subsection{Methodologies}

\blindtext

\subsection{Organization of the CRC research}

\blindtext
\textbf{Stage 1: Classifying relevant phenomena?}

\blindtext


\textbf{Stage 2: Domain-wise theoretical modeling: How should this
  phenomena be modeled in the different cases?}
\blindtext


\textbf{Stage 3: Integrating the World: Towards a more complete,
  integrated model of knowledge.}

\blindtext


\section{Positioning of the planned Collaborative Research Center in
  the academic field}

\begin{erklaerung}
Welche Entwicklungen kennzeichnen national und international das
Forschungsgebiet, auf dem der geplante Sonderforschungsbereich arbeiten
will? Gibt es wissenschaftspolitische oder sonstige Gründe, die eine
verstärkte Förderung des Forschungsgebiets nahe legen? Wie fügt sich der
geplante Sonderforschungsbereich in das Forschungsgebiet ein, inwieweit
geht er über den aktuellen Forschungsstand hinaus?
\end{erklaerung}


\begin{erklaerung}
Welche Forschungsverbünde oder -zentren gibt es auf dem Gebiet des
geplanten Sonderforschungsbereichs in Deutschland (z.B.
Exzellenzcluster, Sonderforschungsbereiche, Forschergruppen,
BMBF-Verbundprojekte) oder im Ausland? Wo gibt es konkurrierende oder
kooperierende Arbeitsgruppen? Wie positioniert sich der geplante
Sonderforschungsbereich gegenüber diesen Aktivitäten?

Wenn innerhalb der letzten drei Jahre aus dem Kreis der
federführend beteiligten Wissenschaftlerinnen und Wissenschaftler heraus
der DFG eine Skizze zur Einrichtung eines Sonderforschungsbereichs mit
einer ähnlichen Fragestellung vorgelegt wurde und daraus keine
Einrichtung resultierte, nennen Sie bitte die wesentlichen Merkmale der
damaligen Initiative/n und erläutern die inhaltlichen, strukturellen und
personellen Unterschiede zur vorliegenden Skizze.
\end{erklaerung}

\section{Research profile of the applicant universities}

\begin{erklaerung}
Welche universitären und außeruniversitären Institutionen sind an
der Initiative beteiligt, welche weiteren Beteiligungen sind geplant?
Gibt es an den beteiligten Institutionen bereits Forschungsschwerpunkte
auf dem Gebiet des geplanten Sonderforschungsbereichs oder ist eine
Schwerpunktbildung geplant? Welche Rolle spielt die Initiative in der
Strukturplanung der antragstellenden Hochschule/n?
\end{erklaerung}


\begin{erklaerung}
Wie ist die personelle Situation an der/den antragstellenden
Hochschule/n? Welche für die Zusammensetzung und das Forschungsprogramm
des Sonderforschungsbereichs wichtigen personellen Wechsel sind erfolgt
oder werden erfolgen? Welche Voraussetzungen sind von Seiten der
antragstellenden oder weiteren beteiligten Institutionen z.B. in Form
von Laboren, Bibliotheken oder Großgeräten gegeben? Wie plant der
Sonderforschungsbereich mit den Forschungsdaten (alle Bezugsquellen und
Ergebnisse des Forschungsprozesses) umzugehen, die im Kontext des
Sonderforschungsbereichs erhoben, ausgewertet und/oder entwickelt
werden?

Ist eine Verstärkung der Grundausstattung durch Personal,
Sachmittel oder Investitionen im Bereich des geplanten
Sonderforschungsbereichs vorgesehen?

Wenn relevant, skizzieren Sie bitte die bestehende oder
angestrebte Vernetzung der beteiligten Institutionen (Hochschulen und
außeruniversitäre Einrichtungen) sowie das Konzept für die
Zusammenarbeit.
\end{erklaerung}

\section{Support structures}

\blindtext
\subsection{Early career support}

\blindtext

\subsection{Gender equality and diversity}

\blindtext

\section{Funding}
\label{sec:funding}

Our research plan requires funding for personnel, guests, workshops,
and events, and special measures for supporting gender equality. An
overview of all funds asked for with this proposal is given in
Table~\ref{tab:reqfund} (entire first funding period) and
Table~\ref{tab:breakfund} (breakdown by subproject).


The group will further benefit from a collaboration with three
internationally renowned specialists who have agreed to serve the group
as Mercator professors, \blindtext

\begin{description}
\item[Birgit Builder (University of There)]
  is at the forefront of a recent trend towards the integration of
  insights and methods from X to Y.
\item[Ben Bender (University of Here)] is a well known
  researcher on individual super differences, further analysis and
  statistics.
\item[Mirjam Muller (University Under Siege)] is
  internationally renowned for her work in and out.
\end{description}



\begin{table*}\centering
%\ra{1.3}
\begin{tabular}{@{}rrlrlrrrrcr}\toprule
  &&&&& Year\,1 & Year\,2 & Year\,3 & Year\,4 && Total \\
\cmidrule{6-9}
\multicolumn{5}{l}{Staff} \\
~~~~& 4 & Post-docs &100\% & E13 & 279.6 & 279.6 & 279.6 & 279.6 &&
   1120 \\
& 7 & Doctoral students & 65\% & E13 &  293.5 & 293.5 & 293.5 & 293.5 &&
   1170\\
% Research assistant commented out by TM. now included in post-docs
%& counts[3] & Research assistant & fracs[3]*100\%  & E13 &  paste(myr(staff[3,]),collapse = " & ") &&
%   myt(sum(staff[3,],na.rm=T))\\
& 1 & Administrative assistant &  50\% & E8 &  24 & 24 & 24 & 24 &&
   100\\
& 4 & Student Workers & \multicolumn{2}{l}{41 hrs/month} &  24 & 24 & 24 & 24 &&
   100\\
\multicolumn{5}{l}{Direct Costs} \\
&& \multicolumn{2}{l}{Travel Costs} && 70 & 65 & 65 & 51 && 
   250 \\
&& \multicolumn{2}{l}{Guests and Conferences} && 88 & 80 & 54 & 63 && 
   280\\
&& \multicolumn{2}{l}{Test subjects etc.} && 15 & 16 & 16 & 10 && 
   60\\
&& \multicolumn{2}{l}{Lump sum} && 73 & 77 & 62 & 87 && 
   300\\
&& \multicolumn{2}{l}{Gender Equality} && 28 & 23 & 23 & 26 && 
   100\\
&& \multicolumn{2}{l}{Replacements} &&  & 47 & 48 & 42 && 
   140\\
&& \multicolumn{2}{l}{Mercator Fellows} && 10 & 10 & 10 & 12 && 
   40\\
\cmidrule{6-9}\cmidrule{11-11}
\multicolumn{3}{l}{Total} & & & 910 & 940 & 900 & 910 & & 3660 \\
\bottomrule
\end{tabular}
\caption{Overview of requested funding. Numbers are in units of \EUR{1000}, totals rounded to \EUR{10000}.}
\label{tab:reqfund}
\end{table*}




\begin{table}[ht]
\centering
\begin{tabular}{lrrrr}
  \toprule
Project & Post-docs (E13) & Doctoral students (65\% E13) & Students & Secretary (50\% E8) \\ 
  \midrule
\PP{A01} & 1.0 & 2.0 & 1.0 &  \\ 
  \PP{B01} &  & 2.0 & 1.0 &  \\ 
  \PP{MGK} & 0.5 & 3.0 & 1.0 &  \\ 
  \PP{INF} & 2.0 &  & 1.0 &  \\ 
  \PP{Z01} & 0.5 &  &  & 1.0 \\ 
   \midrule
Total & 4.0 & 7.0 & 4.0 & 1.0 \\ 
   \bottomrule
\end{tabular}
\caption{ Requested personnel breakdown by subproject (two student assistants in Z01 will be funded by \ABRA)} 
\label{tab:breakfund}
\end{table}



