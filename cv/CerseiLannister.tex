\cvhead{Professor Dr. Cersei Lannister}%

% Die Erklärungen könne, müssen aber nicht gelöscht werden.
% Sie werden in jedem Fall im finalen Dokument zentral deaktiviert.
\erklaer{
Die Forschungsprofile sollen für alle designierten
Teilprojektleiterinnen und -leiter in kurzer Form Angaben zum
beruflichen Lebenslauf und zu Publikationen enthalten. Bitte richten
Sie sich dabei nach der unten stehenden Gliederung. Chronologische
Aufzählungen sollten mit dem am kürzesten zurückliegenden Ereignis
beginnen. Die Erhebung der Geburtsdaten und des Geschlechts der
Teilprojektleiterinnen und Teilprojektleiter dient ausschließlich der
Identitätsfeststellung bzw. statistischen Zwecken.}

% Bitte verwenden Sie die vordefinierten Befehle und Überschriften für
% die Abschnitte im Lebenslauf.

\cvsection{\cvgeninf}
% allgemeine Angaben.
\erklaer{Allgemeine Angaben
\begin{itemize}
\item Name (ggf. auch Geburtsname), Vorname, akademischer Titel,
  Geburtsdatum, Geschlecht
\item  Institutsanschrift, Telefonnummer,
  E-Mail-Adresse
\item Derzeitige Position und Status,
  z.B. Postdoktorand/-in, Emmy-Noether-Nachwuchsgruppenleiter/-in,
  Jun.-Prof., Heisenberg-Stipendiat/-in oder -Professor/-in,
  Professor/-in (C3, C4, W2, W3)
\item Falls familiäre Verpflichtungen wie
  Kinderbetreuung oder die Pflege von Angehörigen den Fortgang Ihrer
  wissenschaftlichen Karriere verzögert haben, geben Sie hier bitte
  die entsprechenden Zeiträume (Elternzeiten, Pflegezeiten) und
  ggf. die Geburtsjahre der Kinder an.
\end{itemize}}

% Ersetzten Sie mit Ihren Angaben.
\cvgi[24 months parenatal leave in total, after the birth of two
children in 2012 and 2017] % Optional. Leer lassen, falls unzutreffend.
{Prof. Dr. Cersei Lannister} % Name mit Titel.
{1972-01-27} % Geburtsdatum YYYY-MM-DD
{f}% f und m funktionieren als Abkürzungen, abweichendes wird übernommen.
{\instn, \unia, \uniaaddress}% Institutsadresse
{+49 30 20931111}% Telefonnummer (bitte Format beibehalten (idealerweise))
{cersei.lannister@unseen-university.uu} % Email
{www.unseen-university.uu/CerseiLannister} % Homepage
{University Professor (W3)} % Position und Status

% Akademische Ausbildung mit Abschluss
\cvsection{\cveduc}
\erklaer{
  Akademische Ausbildung mit Abschluss
  \begin{itemize}
  \item Studienfächer (JJJJ – JJJJ), Universitäten, Abschlüsse,
    Betreuer/-in der Abschlussarbeiten
\end{itemize}
}

\begin{cvtimeline}
  % Das zweite Argument (das Endjahr 2001 im Beispiel) kann leer
  % bleiben.  Die Abkürzung 'now' akzeptiert und sollte statt
  % 'current' oder 'present' verwendet werden.
\cvtlit{1995}{2001}{Philantrophy, Invisible College, Master of Arts,
  Prof. Dr. Meng}
\end{cvtimeline}

%  Wissenschaftliche Abschlüsse
\cvsection{\cvdeg}
\erklaer{
  Wissenschaftliche Abschlüsse
  \begin{itemize}
  \item Habilitation: Fach, Universität, Abschlussjahr, Mentor/-in
  \item Promotion: Fach, Universität, Jahr der mündlichen Prüfung,
    Betreuer/-in
  \item Anderer wissenschaftlicher Abschluss: Fach,
    Universität/Einrichtung, Abschlussjahr, Betreuer/-in oder
    Mentor/-in
  \end{itemize}
}
% Für Promotion und Habilitation bitte die vorgefertigten Makros verwenden.
\begin{cvdesc}
  % There are Macros prepared for Master's degrees and Staatsexamina.
  \cvhabil[The Optional Title]{Philantropy}{Unseen University}{2009}{Prof. Dr. Isaac Newton}
  \cvdoc[The Optional Title]{Philosophy}{Charles University Prague}{2013}{Prof. Dr. Gottfried W. Leibniz}{\drphil}
\end{cvdesc}

% Beruflicher Werdegang ab Studienabschluss
\cvsection{\cvdep}
\erklaer{
  Beruflicher Werdegang ab Studienabschluss
  \begin{itemize}
  \item Zeiten (JJJJ – JJJJ), Position/Funktion (Postdoktorand/-in
    o.ä.), Universität / Einrichtung / Unterneh- men
\end{itemize}
}
\begin{cvtimeline}
  % Auch hier kann das Argument für das Endjahr frei bleiben. Die
  % Abkürzung 'now' akzeptiert und sollte statt 'current' oder
  % 'present' verwendet werden.
  \cvtlit{2002}{now}{Junior Professor, Philippine Institute of
    Volcanology and Seismology}
  \cvtlit{2001}{2002}{Researcher, Institut für
    Kooperationswissenschaft, Universität Atlantis}
  \cvtlit{1998}{2001}{Researcher, Institut für manuelle
    Sprachverwaltung, Universität Überlingen}
\end{cvtimeline}

% Sonstiges
\cvsection{\cvmisc}
\erklaer{
  Sonstiges
  \begin{itemize}
  \item Herausgeberschaften, Funktionen in wissenschaftlichen Beiräten
    oder Beratergremien etc., Auszeichnungen u.a.
  \end{itemize}
}

\begin{cvitemize}
\cvit Member of the Advisory Board of the Journal \textit{My Language}
\cvitft{Referee: DFG, Applied Linguistics, and science foundations in Belgium, France and Poland}{1999}{1999}
\cvitft{The X Price for excellent teaching, University of Y}{2013}{}
\cvitft{Member of the Z Comission}{2011}{now}
\cvitft{Editor of the book series XYZ}{2000}{2005}
\end{cvitemize}

% Selected publications
\cvsection{\cvpub}
\erklaer{
Publikationen.
An dieser Stelle nennen Sie bitte Ihre wichtigsten
Publikationen. Diese müssen nicht im Zusammenhang mit dem von Ihnen
geleiteten Teilprojekt stehen. Sie sind zu gliedern in
}

% peer reviewed:
\peerbib{test1,test3,buch1} % Insert your own bibtex keys here.
% other:
\otherbib{teil1,proc1} % Insert your own bibtex keys here.
% patents:
\patentbib% Output disappears if arguments are empty
  {patent2}%filed
  {patent1}%granted



%%% Local Variables:
%%% mode: latex
%%% TeX-master: "cv"
%%% End:
