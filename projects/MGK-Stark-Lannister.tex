\prhead{MGK}{Integrated graduate school}

\notocsubsection{\pipl}
\begin{giprenv}
  \prgi{Prof.~Dr.~Eddard Stark}
{1965-07-20}
{\instn}
{\unia}
{\uniaaddress}
{(+49) 30 1234 5678}
{ned.stark@university.de}

\end{giprenv}
\gicolsep
\begin{giprenv}
  \prgi{Prof.~Dr.~Cersei Lannister}
{1971-03-03}
{\instn}
{\uniclong}
{\unicaddress}
{(+49) 50 67895432}
{cersei.lannister@university.org}

\end{giprenv}


\notocsubsection{\prsum}
%\notocsubsubsection{\resqu}
% \notocsubsection{Project description}


\notocsubsubsection{Summary}

The main aim of the Integrated Research Training Group \emph{Specific
  Knowledge} (the \PPs{MGK}) is to train a generation of future
scholars in scientific study of special knowledge. Students benefit
from a wide variety of generic and more specialized research training,
supervision by top academics in our field, a stimulating and
extraordinarily broad-based research environment in the thematic area
of the CRC and our science more widely, outstanding opportunities for
informal and more formalized exchange, and excellent facilities.

As a key reference point for PIs, post-docs, and doctoral students
themselves within \thiscrc, the \PPs{MGK} will be a vehicle for
collaboration within the CRC. Thus, it will further the wider
objectives of \thiscrc{} beyond its immediate remit of research
training.  We expect that future doctors trained in the \PPs{MGK} will
move on to occupy key academic positions in Germany and
internationally. Throughout their career, they will take further the
academic agenda of the CRC and ensure that it will leave a lasting
mark in the scholarly community long beyond its period of funding.

\notocsubsubsection{Qualification Program}

The \PPs{MGK} will provide all the ca.~20 doctoral students in
\thiscrc{} with excellent research training and supervision.  There
are three routes of entry: (i) Recruitment directly to CRC projects
for full-length studentships that are built into the project’s
research design and costed in the project. (ii) Central recruitment,
in a yearly open competition on the basis of research potential
considering also the diversity of the CRC, to a pool of \textbf{three}
up to one-year studentships on specific research. Successful
applicants will acquire additional qualifications for specific
research and the most suitable candidates will then change to a
specific project in the CRC to complete their doctorate.  (iii)
Admission as exchange student for shorter periods, normally in the
context of a fellowship (see below).  Apart from the different mode of
recruitment and admission, training, supervision, and support
provisions are identical for cohorts (i), (ii), and (iii), but apply
pro-rata to non full-length students.


Each student will be attached to at least one designated project within \thiscrc. It is expected that each student has two designated supervisors, one of whom should normally be a PI or post-doc from the project the student is attached to, and the other one from a different project, to strengthen  cohesion and to encourage cross-fertilization within the CRC. The second supervisor may also be an academic from \ABRA{} or \ABRB{} with no formal role in the CRC. Supervision arrangements will be discussed with the \PPs{MGK} leadership. At the beginning of their doctoral studies, supervisors and student will identify each student’s training needs and agree the best arrangements to meet these needs, as detailed below. In specific cases of students with M.A.s from institutions other then \ABRA{}, some research training needs as identified by supervisors may lead to students taking relevant modules from the masters’ program offered at \ABRA{}, as appropriate. Progress and arrangements will be reviewed at the beginning of the second, third, and any subsequent year. Via frequent feedback, we make sure that the wealth of opportunities that our students enjoy facilitates timely completion of their degree.



Research training for doctoral students has a generic and a more individual aspect. Currently both \ABRA{} and \ABRB{} offer some generic skills training for doctoral students, some of which is delivered under the umbrella of Our Graduate School, but  no structured training program or specific advanced training for doctoral students in our field.  The large number of doctoral students in \thiscrc{} and the focus on specific knowledge creates the opportunity to offer a structured program tied to the needs of the CRC. We place great emphasis on training in research methods and advanced training specific to the CRC. We recognize that the quality of doctoral education depends not only on direct supervision and tuition, but also on broader arrangements, including student peer-to-peer input.

Specifically, we will offer the following seven training elements:

%\begin{center}
%\begin{tabular}{ccc}
%measure&period&details\\\hline
%method school&yearly&\\
%CRC colloquium&weekly&\\
%PhD peer-to-peer&yearly&\\
%\end{tabular}
%\end{center}
%

\begin{enumerate}
\item\textbf{Methods and theory schools:} We will offer a dedicated
  research training module for \PPs{MGK} students on research in our field.
  Some of the content covered will include: foo and bar.
\item \textbf{Individualized Training:} CRC staff, whether PIs or
  researchers, will make themselves available to dispense more
  individualized training. Where appropriate, individual training
  needs may also be met outside the CRC. Students will benefit from an
  individual allowance that will enable them, with the consent of
  their supervisor, to attend conferences or specialized training
  events outside the CRC. In addition, the \PPs{MGK} has a generous budget
  for student fieldwork.
\item \textbf{Study abroad} To help students build an international
  network and to broaden the diversity of the student body, we
  advertise annually one-year funded visiting fellowship positions for
  international doctoral students to visit the CRC.  We also plan to
  offer our students the opportunity to study and carry out research
  at a partner institution for defined periods (from one month
  upwards) where it is appropriate and beneficial for their primary
  project. We have good informal contacts with a lot of universities.

\item \textbf{CRC Colloquium} There will be a weekly \thiscrc{}
  research colloquium. All doctoral students should present and discuss their
  project at least once at this colloquium. \item \textbf{Mentor/CRC
    guest} The \PPs{MGK} has a budget that will allow each student to invite
  an international researcher working in the area of their doctoral project
  to attend the student’s presentation and to be their discussant. We
  encourage the student and their guest to develop this arrangement
  into a more long-term mentor-mentee relationship.
\item \textbf{Doctoral peer-to-peer retreat} On a yearly working
  retreat, the doctoral students (open to post-docs at the discretion
  of the doctoral students) will discuss with each other the
  challenges they face with their research, and strategies to solve
  these problems. They will also build a network within the CRC for
  scientific and emotional support to aid the completion.
\item \textbf{Gathering} Monthly meeting of the doctoral students
  (open to post-docs at the discretion of the doctoral students). This
  forum will provide an opportunity to discuss ongoing student work,
  any technical issues of significance for the \PPs{MGK} student body,
  and debate on current issues in specific research.
\end{enumerate}




\notocsubsubsection{Management and Supervision}

The \PPs{MGK} is led by PIs Stark (\ABRB{}) and Lannister (\ABRA{}). Both report directly to the spokespeople of \thiscrc. While ultimate responsibility for supervision arrangements, student progress, and examination lies with the relevant bodies of the degree-awarding institution (\ABRA{}) as defined by the doctoral regulations (Promotionsordnung) of \ABRA{}’s Faculty of Our Field, Stark and Lannister have general oversight of the delivery of the qualification program. They work closely with supervisors, most of whom will be PIs or project post-docs, in order to ensure that students’ projects benefit from, or are directly integrated in, the CRC-funded research. Furthermore, the \PPs{MGK} will be affiliated with \ABRA{}'s Graduate School.

Great care will be taken to ensure that arrangements at the CRC and \PPs{MGK} work to mutual benefit, while recognising and paying due attention to  the natural differences in aims and objectives of the two structures. To ensure smooth cooperation, \PPs{MGK} leaders will sit on the CRC’s executive board. They will work closely with supervisors, PIs, relevant bodies, and committees of \ABRA{}’s Faculty of Our Field to identify opportunities and how to put them to use, as well as addressing any issue or matter of concern.

The doctoral students will elect one representative and an alternate to work together with the PIs to represent the doctoral students' interest within the CRC.  The PIs will meet regularly with the doctoral representatives and the doctoral representative is integrated into the executive structure of the CRC.

The \PPs{MGK} will be a central nexus of the CRC research agenda.  The work of the \PPs{MGK} will be coordinated by a designated research coordinator working 50\% of full time.  They will assist the PIs in managing the qualification program, tracking progress toward the doctorate, and more generally ensuring the day-to-day operation of the \PPs{MGK}.

\notocsubsubsection{Environment of the Integrated Research Training Group}

Doctoral students will benefit from an exceptionally rich and
stimulating research environment at \ABRA{} and \ABRB{}. Our Field at
\ABRA{} is among the leading institutions in the country. \ABRB{}, is
great as well.

Every effort will be made to offer doctoral students a dedicated
common room within the CRC’s estate. Likewise, we will ensure that
doctoral students are offered adequate workspace in their project.

The \PPs{MGK} is an important contributor to the \textbf{diversity} of
\thiscrc.  We will strive to attract minority students to the CRC. The
one-year fellowship will enable us to attract students from the widest
possible talent pool. The international students supported by the CRC
contribute to the diversity of the CRC/\PPs{MGK} constituency.

\notocsubsection{\funds}

We request funding for 1 50\% post-doc research associate as \PPs{MGK}
manager and 3 doctoral positions for doctoral research fellows.

