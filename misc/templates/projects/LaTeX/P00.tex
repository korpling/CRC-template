\prhead{A01}{This Project has a Title}

\notocsubsection{\pisg}

\begin{erklaerung}
Explanations in German copied from the official documents. %


Bitte geben Sie an: Akademischer Titel, Vorname, Nachname, Geburtsdatum, Institutsanschrift, Telefonnummer, E-Mail-Adresse.
\end{erklaerung}

\begin{giprenv}
\prgi{Prof.~Dr.~Eddard Stark}
{1965-07-20}
{\instn}
{\unia}
{\uniaaddress}
{(+49) 30 1234 5678}
{ned.stark@university.de}
\end{giprenv}

\notocsubsection{\prsum}
\begin{erklaerung}
  Bitte skizzieren Sie: zentrale wissenschaftliche Fragestellung,
  Forschungsstand, projektrelevante eigene Vorarbeiten,
  Arbeitsprogramm und Methodik, Einbindung in den
  Sonderforschungsbereich.

  Die Angaben in diesem Kapitel sollen in sich geschlossen sein,
  d.h. auch ohne die Lektüre zusätzlicher Dokumente verständlich,
  schlüssig und beurteilbar. Es sind nur Arbeiten zu zitieren, deren
  Resultate und Methoden inhaltlich diskutiert werden. Bei der
  Zitation eigener Veröffentlichungen soll auf die \enquote{Liste
    projektrelevanter eigener Publikationen} im Abschnitt
  \ref{sec:ownpub} verwiesen werden. Publikationen anderer, nicht am
  geplanten Sonderforschungsbereich beteiligter Autoren, die für das
  Vorhaben wichtig sind, können in einem Verzeichnis z.B. am Ende des
  Abschnitts \enquote{Stand der Forschung} aufgelistet werden.  In
  dieses Verzeichnis können falls notwendig noch andere Dokumente
  aufgenommen werden. Sollte es sich dabei um nicht publizierte
  Arbeiten handeln, sind diese in der PDF-Datei \enquote{Nicht
    öffentlich zugängliche Texte} auf der CD mit \enquote{Unterlagen
    für die Geschäftsstelle} beizufügen. Bitte beachten Sie, dass die
  Lektüre weiterer Dokumente durch die Beratungsgruppe in jedem Falle
  optional ist, Bewertunggrundlage ist ausschließlich die
  vorgelegte Skizze.  Folgende Untergliederung kann in diesem Kapitel
  verwendet werden:
\nocite{test2}
\end{erklaerung}
\notocsubsubsection{\resqu}

\blindtext \textcite{teil1}

\notocsubsubsection{\state}

\blindtext \textcite{test3}

\notocsubsubsection{\respr}

\paragraph{\resprp}

\blindtext

\paragraph{\role}

\blindtext

% peer reviewed:
\peerbib{test1,test3,buch1} % Insert your own bibtex keys here.
% other:
\otherbib{teil1,proc1} % Insert your own bibtex keys here.
% patents:
\patentbib% Output disappears if arguments are empty
  {patent2}%filed
  {patent1}%granted
%% In case there is only peer reviewed literature to cite, just use:
% \prbib{test2,buch1,teil1}
\label{sec:ownpub}

\begin{erklaerung}
  An dieser Stelle sind ausschließlich eigene Arbeiten
  \parencite{test1} zu nennen, die in direktem inhaltlichen
  Zusammenhang mit dem vorgeschlagenen Teilprojekt stehen und
  öffentlich zugänglich gemacht wurden. Sie sind zu gliedern in
  \begin{enumerate}
    \renewcommand{\theenumi}{\alph{enumi}}
  \item \label{it1} Arbeiten, die in Publikationsorganen mit einer
    wissenschaftlichen Qualitätssicherung zum Zeitpunkt der Erstellung
    der Skizze erschienen oder endgültig angenommen sind, und
    Buchveröffentlichungen;
  \item \label{it2} Andere Veröffentlichungen; und
  \item Patente, gegliedert nach angemeldet und erteilt.
  \end{enumerate}
  Die Zahl der Nennungen ist in \ref{it1} und \ref{it2} zusammen auf
  zehn begrenzt. Wenn zur Publikation angenommene, aber noch nicht
  erschienene Arbeiten aufgelistet werden, sind diese zusammen mit
  einem datierten Beleg der Annahme in der PDF-Datei \enquote{Nicht
    öffentlich zugängliche Texte} auf der CD mit \enquote{Unterlagen
    für die Geschäftsstelle} beizufügen. Die Geschäftsstelle stellt
  der Beratungsgruppe eine elektronische Kopie der Skizze sowie die
  PDF-Datei mit nicht öffentlich zugänglichen Texten über das elan-
  Portal bereit.
\end{erklaerung}

\notocsubsection{\funds}

We request funding for 2 doctoral student positions, one for a
researcher with statistical knowledge and one for a researcher with
experience in conducting experiments.

\begin{erklaerung}
  Wird das Projekt derzeit durch die DFG oder eine andere Institution
  gefördert? Wie stellt sich der erwartete Bedarf an Personal und
  größeren Geräten dar, für den Mittel im geplanten
  Sonderforschungsbereich beantragt werden sollen?
\end{erklaerung}
